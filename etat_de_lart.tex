\section{État de l'art}

% slide detection de vehicule
\subsection{Détection du véhicule}
\begin{frame}
\frametitle{État de l'art}

Détection du véhicule. Méthode ``naïve'':
\begin{itemize}
    \item recherche de caractéristiques bas-niveau (ombres, arêtes, symétries…)
    \item peu fiable avec beaucoup de faux-positifs et sensible aux variations lumineuses
    \item rapide et utilisée pour présélectionner des zones de recherche
\end{itemize}
\end{frame}

\begin{frame}
\frametitle{État de l'art}

Détection du véhicule. Méthode ``lourde'':
\begin{itemize}
    \item machine learning (SVM et NN) sur l'ensemble de l'image avec des descripteurs (orientation des arêtes, textures…)
    \item lourd en calculs
    \item utilisée après une présélection de zones
\end{itemize}
\end{frame}

% slide detection de marquage au sol
\subsection{Détection du marquage au sol}
\begin{frame}
\frametitle{État de l'art}

Détection de marquage au sol:
\begin{itemize}
    \item technique la plus étudiée car indispensable au guidage latéral
    \item nombreux problèmes de détection (ombres, variations de contraste, marquage effacé, obstruction par véhicule…)
    \item détection de zones où des lignes peuvent être présentes (détection de Canny)
    \item application d'un modèle mathématiques (ligne droite, quadratique…)
    \item tracking pour accélérer le temps de calcul et la continuité
\end{itemize}

\end{frame}

% slide combinaison des 2 methodes ci-dessus
\subsection{Combinaison des 2 méthodes précédentes}
\begin{frame}
\frametitle{État de l'art}

Combinaison des 2 techniques précédentes:
\begin{itemize}
    \item utiliser les 2 techniques séparément et combiner leurs résultats
    \item utiliser le marquage pour assister le processus de détection de véhicule
    \item utiliser les 2 techniques simultanément pour ``s'auto-optimiser''
\end{itemize}

\end{frame}
